\documentclass[letterpaper,10pt,titlepage]{article}

\usepackage{graphicx}

\usepackage{amssymb}
\usepackage{amsmath}
\usepackage{amsthm}

\usepackage{alltt}
\usepackage{float}
\usepackage{color}
\usepackage{verbatim}

\usepackage{geometry}
\geometry{textheight=10in, textwidth=7.5in}

\usepackage{hyperref}

\def\name{Vlad Predovic}

%pull in the necessary preamble matter for pygments output
\input{pygments.tex}

%% The following metadata will show up in the PDF properties
\hypersetup{
  colorlinks = true,
  urlcolor = black,
  pdfauthor = {\name},
  pdfkeywords = {cs311 ``operating systems'' files filesystem I/O},
  pdftitle = {CS 311 Project 3: UNIX Process Control},
  pdfsubject = {CS 311 Project 3},
  pdfpagemode = UseNone
}

\parindent = 0.0 in
\parskip = 0.2 in

\begin{document}
\section{Design}

\begin{itemize} 
\item The design of my program will begin by taking into consideration everything required from the algorithm in the compute.c file. 
Doing as much of compute.c first will help me to structure how I will need to send and recieve information.

\item For this assignment I really want to focus on testing everything locally before beginning to connect the many parts
of my project. For this reason I next plan on implementing the timing loop in my compute.c file before adding in communication.
I will insert rudimentary versions of the client/server examples into the manage/compute files respectively while building the other areas of the
assignment.

\item Next I will look to implement a simple server on my manage.py file which upon completion, will be morphed
into a concurrent server.

\item The parsing will come last since I can implement simple communication 
without detecting whether or not I want to kill the process or compute the files.
\ldots 
\end{itemize}



\subsection{Where I differed}
For testing purposes I ended up using a server written in C, using many of the details provided by the walkthrough in the link below.
This was due to the fact that I programmed most compute.c first, except for the parsing and threading sections.
http://www.linuxhowtos.org/C_C++/socket.htm?userrate=1


\section{Work log}
\verbatiminput{log.txt}

\section{Challenges}
My main concern was conceptually grasping the relationship between created child processes 
and how to correctly implement piping between my newly created processes.


\section{Answers}
\subsection{Main point of the assignment}
Learn the concepts of forking and piping. The basics of how a shell and system() function.
Work with signals 

\subsection{Solution Accuracy}
My Shell runs correctly when tested with the implemented functions. You can access files, 
change directories, and run programs. However, I had to resort to using execvp as I was having major issues
:( .


\subsection{Learnings}
Grasped a better concept of the basis of a shell and how to create one. How they convert the information 
into action and deal with running different processes. Initially I had no idea about the whole child creation process.

\end{document}