\documentclass[letterpaper,10pt,titlepage]{article}

\usepackage{graphicx}

\usepackage{amssymb}
\usepackage{amsmath}
\usepackage{amsthm}

\usepackage{alltt}
\usepackage{float}
\usepackage{color}
\usepackage{verbatim}

\usepackage{geometry}
\geometry{textheight=10in, textwidth=7.5in}

\usepackage{hyperref}

\def\name{Vlad Predovic}

%pull in the necessary preamble matter for pygments output
\input{pygments.tex}

%% The following metadata will show up in the PDF properties
\hypersetup{
  colorlinks = true,
  urlcolor = black,
  pdfauthor = {\name},
  pdfkeywords = {cs311 ``operating systems'' files filesystem I/O},
  pdftitle = {CS 311 Project 3: UNIX Process Control},
  pdfsubject = {CS 311 Project 3},
  pdfpagemode = UseNone
}

\parindent = 0.0 in
\parskip = 0.2 in

\begin{document}
\section{Design}
	My initial design was to copy sections of an array into different threads or processes and have it compile the results after all processes were
	completed. Although it worked relatively well with low intervals, I quickly understood this design was flawed so I reset and started over.
	After discussing with peers and the instructor, i redesigned my system so i could initally feed my primes into a bit mask. I used chars to mask the bits,
	and I was able to fit four numbers per char as I wanted to leave a bit for designating the happy/sad state of the number. 
	My plan was to mask my bit array in a file of this format and then feed it through the threads/processes which would each be given offsets depending on the amount.
	
	This part was difficult until i implemented a global array which simplified alot of the copying I was previously doing. 
	I used primarily the examples from class and man pages to apply threads and processes. In order to handle signals I used the previous assignments work and revisited 
	the textbook.
\section{Work log}
\verbatiminput{log.txt}

\section{Challenges}
My greatest difficult dealt with memory allocation. I learned a lot having to manipulate different types in order to efficiently pass my data
along to different processes. I struggled greatly with even getting my algorithm to give me a bit array with uint entries.


\section{Answers}
\subsection{Main point of the assignment}
More programming at the system level. For me the most important port of this assignment was learning to manipulate
bits and functionality(the threads and processes) when dealing with heavy loads on memory and process time. I feel this
was a critical skill to pick up although I have not seen the sun in the last week because of it.

\subsection{Solution Accuracy}
I used a variety of primitive testing methods, primarily printing the many iterators I used to keep track of my variables.
I am confident that my solution works correctly as the list of Happy Primes is not that extensive and creating a simple algorithm
to print my results showed that my answers were in line with expected results for the first few thousand numbers.

Also the command xxd -b was great for checking my low numbers. It prints the contents of a file in binary to the command line.

\subsection{Learnings}
My understanding of how to manipulate memory and make programs much more efficient has increased greatly.
I feel this is a very important set of tools to keep in mind when building programs to be memory-efficient and short on run-time.

\end{document}