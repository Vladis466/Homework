\documentclass[letterpaper,10pt,titlepage]{article}

\usepackage{graphicx}

\usepackage{amssymb}
\usepackage{amsmath}
\usepackage{amsthm}

\usepackage{alltt}
\usepackage{float}
\usepackage{color}
\usepackage{verbatim}

\usepackage{geometry}
\geometry{textheight=10in, textwidth=7.5in}

\usepackage{hyperref}

\def\name{Vlad Predovic}

%pull in the necessary preamble matter for pygments output
\input{pygments.tex}

%% The following metadata will show up in the PDF properties
\hypersetup{
  colorlinks = true,
  urlcolor = black,
  pdfauthor = {\name},
  pdfkeywords = {cs311 ``operating systems'' files filesystem I/O},
  pdftitle = {CS 311 Project 3: UNIX Process Control},
  pdfsubject = {CS 311 Project 3},
  pdfpagemode = UseNone
}

\parindent = 0.0 in
\parskip = 0.2 in

\begin{document}
\section{Design}
The design for my program consisted of three separate parts. The first take was to create a parser
to interact with the user and the rest of the program. The next was to develop my built in commands.
For my simple Shell I wrote a 'help' menu to make grading easier. The commands ls, cd, and exit are 
included. Finally the third task was to avoid having to access the binsh shell and implement my own
to execute the commands.
For the most part I followed this design. There weren't any aggressive changes I had to make to the overall
scope of the design
\section{Work log}
\verbatiminput{log.txt}

\section{Challenges}
My main concern was conceptually grasping the relationship between created child processes 
and how to correctly implement piping between my newly created processes.


\section{Answers}
\subsection{Main point of the assignment}
Learn the concepts of forking and piping. The basics of how a shell and system() function.
Work with signals 

\subsection{Solution Accuracy}
My Shell runs correctly when tested with the implemented functions. You can access files, 
change directories, and run programs. However, I had to resort to using execvp as I was having major issues
:( .


\subsection{Learnings}
Grasped a better concept of the basis of a shell and how to create one. How they convert the information 
into action and deal with running different processes. Initially I had no idea about the whole child creation process.

\end{document}