\documentclass[letterpaper,10pt,titlepage]{article}

\usepackage{graphicx}                                        

\usepackage{amssymb}                                         
\usepackage{amsmath}                                         
\usepackage{amsthm}                                          

\usepackage{alltt}                                           
\usepackage{float}
\usepackage{color}

\usepackage{url}

\usepackage{balance}
\usepackage[TABBOTCAP, tight]{subfigure}
\usepackage{enumitem}

\usepackage{pstricks, pst-node}

\usepackage{geometry}
\geometry{textheight=10in, textwidth=7.5in}

\usepackage{listings}

\definecolor{mygreen}{rgb}{0,0.6,0}
\definecolor{mygray}{rgb}{0.5,0.5,0.5}
\definecolor{mymauve}{rgb}{0.58,0,0.82}

\lstset{ %
  basicstyle=\ttfamily,            % the size of the fonts that are used for the code
  breakatwhitespace=false,         % sets if automatic breaks should only happen at whitespace
  breaklines=true,                 % sets automatic line breaking
  captionpos=b,                    % sets the caption-position to bottom
  commentstyle=\color{mygreen},    % comment style
  escapeinside={\%*}{*)},          % if you want to add LaTeX within your code
  extendedchars=true,              % lets you use non-ASCII characters; for 8-bits encodings only, does not work with UTF-8
  keepspaces=true,                 % keeps spaces in text, useful for keeping indentation of code (possibly needs columns=flexible)
  keywordstyle=\color{blue},       % keyword style
  numbers=left,                    % where to put the line-numbers; possible values are (none, left, right)
  numbersep=10pt,                  % how far the line-numbers are from the code
  numberstyle=\tiny\color{mygray}, % the style that is used for the line-numbers
  rulecolor=\color{black},         % if not set, the frame-color may be changed on line-breaks within not-black text (e.g. comments (green here))
  showspaces=false,                % show spaces everywhere adding particular underscores; it overrides 'showstringspaces'
  showstringspaces=false,          % underline spaces within strings only
  showtabs=false,                  % show tabs within strings adding particular underscores
  stepnumber=1,                    % the step between two line-numbers. If it's 1, each line will be numbered
  stringstyle=\color{mymauve},     % string literal style
  tabsize=8,                       % sets default tabsize to 8 spaces
  %title=\lstname                  % show the filename of files included with \lstinputlisting; also try caption instead of title
}

\lstdefinestyle{niceC}{
  belowcaptionskip=1\baselineskip,
  breaklines=true,
  frame=L,
  xleftmargin=0.2in,
  language=C,
  showstringspaces=false,
  basicstyle=\footnotesize\ttfamily,
  keywordstyle=\bfseries\color{green!40!black},
  commentstyle=\itshape\color{purple!40!black},
  identifierstyle=\color{blue},
  stringstyle=\color{orange},
}

%random comment

\newcommand{\cred}[1]{{\color{red}#1}}
\newcommand{\cblue}[1]{{\color{blue}#1}}

\usepackage{hyperref}

\def\name{Vlad Predovic}

%% The following metadata will show up in the PDF properties
\hypersetup{
  colorlinks = true,
  urlcolor = black,
  pdfauthor = {\name},
  pdfkeywords = {cs344 ``operating systems'' files filesystem I/O},
  pdftitle = {Comparison of OS API's},
  pdfsubject = {CS 344 Final},
  pdfpagemode = UseNone
}

\parindent = 0.0 in
\parskip = 0.2 in

\begin{document}
\title{Comparison of selected POSIX and Windows API \\ CS 344 Final, summer 2015}
\author{Vlad Predovic}
\maketitle

\section{Introduction}
	Throughout this class several key topics of system programming have covered. Among these are file I/O, shell implementation with piping and redirection, 
distributing work through threading or spawning sub-processes, and cross language server/client interaction. However, 
if there is any key lesson to be learned about systems programming from this class it is just how quickly, and obscurely, 
things can go wrong if system level implementations are not handled with meticulous scrutiny(synonyms x \= 2  for emphasis).

	The following three sections provide a brief summary of the importance behind each tool. 
A comparison of differences in the API implementation in POSIX and windows will then be laid out.
Finally, these differences will be discussed, and, as in most duels to the death, a choice will be made for better implementation.
The topics covered will be socket implementation(specifically network sockets), file I/O, and communication through Piping.


\section{Sockets:}
	Sockets were first implemented in UNIX as a method allowing autonomous processes to communicate. Most socket implementations
at their core follow a client-server model where the server creates a socket that 'listens'
while the client creates a socket which attempts to to connect to the former. Ultimately, information can
be sent back and forth until the connection is interrupted.


\begin{enumerate}
  \item Set-up
Header files used: Windows(winsock.h)			POSIX(sys/socket.h, arpa/inet.h, netinet/in.h)

	For the most part, set-up of a simple socket connection is identical. Both implementations use the 'sockaddr\_in'
struct to maintain addressing data for create the connection. However, the windows implementation requires a few extra lines
to initialize by using the WSAStartup() function. As a result, information on the WinSock implementation is stored in the WSADATA
struct and the function serves to initialize much of the data and structures that are necessary to use the sockets.

  \item Main Functionality \newline
	Both the windows API and POSIX standard use the same basic methods and calls to communicate between
client and server.


  \item Clean up and Error Checking \newline 
  

	Since we initialize an additional struct we have to close it as well using WSAcleanup(). As for error 
checking, POSIX relies on the errno library to set flags and report while the windows API uses 
WSAgetlasterror() which reports the last error that occurred. Instead of directly closing the socket by
accesing the file descriptor using close(), we use the close-socket() function.

\end{enumerate}
	The changes are slight between the two implementations. A few different function calls are used to 
initialize, clean up, and error check functionality with basically the same parameters and functionality. These
basically consist of prefixing POSIX function call names with 'WSA'(ex: WSAsend). Windows is a little cleaner 
with only one necessary header file, and the WSAData struct can be ignored for the most part unless the user 
specifically wants information it contains. However, windows 8 introduced specific socket API's like WinRT 
and WebSockets, extensions used for specific purposes like high performance for developing network apps and special bi-directional web sockets.

Although very similar, Windows deserves a slight edge due to having additional and more updated functionality
which can be used for very specific purposes.


\section{File I/O}
	One of the most critical aspects of an operating system is how objects interact with 
each other and the user. File I/O(input and output) refers to the operations used to access,
read, write, display, and manipulate standard files within the OS. 
	
\begin{enumerate}
  \item Header files used: Windows(alot....)\newline			POSIX(sys/stat.h, fcntl.h, sys/types.h)
  
  \item Main Functionality \newline
	OS functionality is to an extent defined by the fundamental needs and performance considerations 
a working system as we know it. Due to this, many similarities can be found between the operations
in the Windows and POSIX implementations. The CreateFile() call is the windows equivalent of the POSIX open()
call. It also returns a file descriptor as an integer used to refer to files such as sockets/terminals/etc..
In both versions a user must enter a pathname, access flags, and permissions. These similarites are seen throughout
most of the function calls, at least on a superficial levels. How both operating systems achieve the similar results 
tends to vary a little. 	

  \item File Descriptors(FD) \newline
I decided to cover file descriptors as these are critical to manipulating files at a systems level.
On both POSIX and Windows FD's map to their respective descriptions through a table. 
However, FD objects in POSIX can be sent via sockets while in windows they cannot. Windows instead
employs the function DuplicateHandle() which can be called by both the source and target processes. That
way when messages are received over a socket, the FD is hashed into the table of the current process.

  \item Linearity \newline
	One of the great things about UNIX had to do with the term 'everything is a file'. The Linux kernel 
represents information from applications as if they were files, which makes them able to be accessed by 
the same functions that we use to garner information from a regular text file. Windows on the other hand
stores important details like the registry and system logs as databases. However, you cannot just access 
these like normal files. In windows, many different versions of read(), write(), and other fundamental tools
are necessary to access the different file types.

\end{enumerate}
 Although one could argue that Windows can provide much more specialized tool sets which can give extra
information, for the task of systems programming POSIX seems more powerful. From this brief comparison,
 the POSIX implementation is clearly superior as it grants you much more access and control
with far fewer tools.


\section{Threading} 
	Threading is the process by which we divide a set of tasks into smaller tasks, each with its own execution instructions, 
and run through these all semi-concurrently. On multi core processors, these threads are capable of running in parallel.
If implemented correctly, threading can increase speed up your program significantly. 

\begin{enumerate}
  \item Set-up \newline
Header files used: Windows(process, thread)			POSIX(signal.h, pthread.h)
	
	In Windows all thread types belong to the object HANDLE. This simplifies implementation as opposed to 
the POSIX implementation where each objects can have different data types (pthread\_t, pthread\_cond\_t). 
OIn Windows you use system calls to work with threads while in POSIX we have implemented a library.


  \item Main Functionality \newline
	In both Windows and and POSIX threads are passed one parameter and return a single value. In windows
this value is of type DWORD. In addition, in Windows the kernel handles the execution of all threads. 
Since Windows only uses one object, the amount of functions needed is greatly reduced
and we get some increased flexibility kind of like file I/O with POSIX. A great example is the WaitForMultipleObjects function call.
This handle can be set to any combination of threads, mutexes, semaphores, and events, granting us functionality that
would be much trickier to achieve with POSIX thread implementations. M

  \item Resource Accessibility \newline
Both POSIX and Windows also have trouble dealing with resource accessibility. While both POSIX
and WINDOWS implement semaphores and mutexes to deal with this, Windows also uses events and critical sections while POSIX 
conditional variables. One of the advantages Windows provides its users with is the conditional variable. This
allows synchronization based on the value of data instead of just focusing on controlling the access
to data that threads have.



\end{enumerate}

	I would have to say that the implementation of threads by Windows is more appealing since we are able to encapsulate
all the different types within the same object. This ultimately makes the uses quicker to learn threading implementations.
In addition, the use of implementations like the conditional variable provide flexibility which is not available in the
POSIX implementation


\section{Piping}
\begin{enumerate} 
	\item Useage \newline
	Piping is a method that has been critical to the communication between processes throughout the development and usage
	of UNIX systems. Windows also implements piping for interprocess communication  with the same first-in first-out (FIFO)
	properties. Each pipe has an entry and exit, through which the read and write respectively. These are known as handles in Windows.
	
	\item Comparison \newline
	Pipes in POSIX are basically used by having another program transmit data. In Windows similar methods are used. An interesting
	feature built in to Windows is the use of Mailslots. A Mailslot is used for one way intercommunication. Messages can be stored there,
	and then the owner of the Mailslot can at some point come and retrieve the message. This is similar and is implemented in client-server
	socket systems through windows. Another feature available are named pipes. Named pipes are similar to traditional pipes except they 
	can communicate with multiple processes that do not need to necessarily exist at the same time.
	
	To my knowledge both implementations are fairly similar, at least from a high-level perspective. However,
	the idea of mailslots highly appeals to me, particularly because I would want to try it with the last assignment we 
	did using a concurrent server and transmitting socket information.
	
\end{enumerate}

\section{Conclusion}

	I have now provided four brief comparisons of API implementations in POSIX in Windows. I did not expect
the implementations to be as comparable as they were, to the point where simple implementations of certain functions,
like a client-server connection, are practically identical. Each implementation has its own advantages, but ultimately 
the Windows implementation appears to be more complex and customizable and ultimately more useful at the moment due to
the enormous quantity of software that runs on Windows.



	
\section{Primary Sources}

\href{https://msdn.microsoft.com/en-us/library/ms740525(v=vs.85).aspx }{'Socket Options.' (Windows). N.p., n.d. Web. 14 Aug. 2015.}

\href{https://msdn.microsoft.com/en-us/library/windows/desktop/ms740642(v=VS.85).aspx }{"What's New for Windows Sockets." (Windows). N.p., n.d. Web. 14 Aug. 2015.}

\href{http://man7.org/linux/man-pages/man2/socket.2.html }{"Socket(2) - Linux Manual Page." Socket(2) - Linux Manual Page. N.p., n.d. Web. 14 Aug. 2015.}

\href{http://yarchive.net/comp/linux/everything_is_file.html }{"The Everything-is-a-file Principle (Linus Torvalds)." The Everything-is-a-file Principle (Linus Torvalds). N.p., n.d. Web. 14 Aug. 2015.}

\href{http://man7.org/linux/man-pages/man2/open.2.html }{"Open(2) - Linux Manual Page." Open(2) - Linux Manual Page. N.p., n.d. Web. 14 Aug. 2015.}

\href{http://linux.die.net/man/7/pthreads }{"Pthreads(7) - Linux Man Page." Pthreads(7): POSIX Threads. N.p., n.d. Web. 15 Aug. 2015.}

\href{https://msdn.microsoft.com/en-us/library/windows/hardware/ff565772(v=vs.85).aspx}{"Windows Kernel-Mode Process and Thread Manager." - Windows 10 Hardware Dev. N.p., n.d. Web. 15 Aug. 2015.}

\href{https://msdn.microsoft.com/en-us/library/windows/desktop/ms682516(v=vs.85).aspx }{"Creating Threads." (Windows). N.p., n.d. Web. 15 Aug. 2015.}
\end{document}
