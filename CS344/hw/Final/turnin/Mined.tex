\documentclass[letterpaper,10pt,titlepage]{article}

\usepackage{graphicx}                                        

\usepackage{amssymb}                                         
\usepackage{amsmath}                                         
\usepackage{amsthm}                                          

\usepackage{alltt}                                           
\usepackage{float}
\usepackage{color}

\usepackage{url}

\usepackage{balance}
\usepackage[TABBOTCAP, tight]{subfigure}
\usepackage{enumitem}

\usepackage{pstricks, pst-node}

\usepackage{geometry}
\geometry{textheight=10in, textwidth=7.5in}

\usepackage{listings}

\definecolor{mygreen}{rgb}{0,0.6,0}
\definecolor{mygray}{rgb}{0.5,0.5,0.5}
\definecolor{mymauve}{rgb}{0.58,0,0.82}

\lstset{ %
  basicstyle=\ttfamily,            % the size of the fonts that are used for the code
  breakatwhitespace=false,         % sets if automatic breaks should only happen at whitespace
  breaklines=true,                 % sets automatic line breaking
  captionpos=b,                    % sets the caption-position to bottom
  commentstyle=\color{mygreen},    % comment style
  escapeinside={\%*}{*)},          % if you want to add LaTeX within your code
  extendedchars=true,              % lets you use non-ASCII characters; for 8-bits encodings only, does not work with UTF-8
  keepspaces=true,                 % keeps spaces in text, useful for keeping indentation of code (possibly needs columns=flexible)
  keywordstyle=\color{blue},       % keyword style
  numbers=left,                    % where to put the line-numbers; possible values are (none, left, right)
  numbersep=10pt,                  % how far the line-numbers are from the code
  numberstyle=\tiny\color{mygray}, % the style that is used for the line-numbers
  rulecolor=\color{black},         % if not set, the frame-color may be changed on line-breaks within not-black text (e.g. comments (green here))
  showspaces=false,                % show spaces everywhere adding particular underscores; it overrides 'showstringspaces'
  showstringspaces=false,          % underline spaces within strings only
  showtabs=false,                  % show tabs within strings adding particular underscores
  stepnumber=1,                    % the step between two line-numbers. If it's 1, each line will be numbered
  stringstyle=\color{mymauve},     % string literal style
  tabsize=8,                       % sets default tabsize to 8 spaces
  %title=\lstname                  % show the filename of files included with \lstinputlisting; also try caption instead of title
}

\lstdefinestyle{niceC}{
  belowcaptionskip=1\baselineskip,
  breaklines=true,
  frame=L,
  xleftmargin=0.2in,
  language=C,
  showstringspaces=false,
  basicstyle=\footnotesize\ttfamily,
  keywordstyle=\bfseries\color{green!40!black},
  commentstyle=\itshape\color{purple!40!black},
  identifierstyle=\color{blue},
  stringstyle=\color{orange},
}

%random comment

\newcommand{\cred}[1]{{\color{red}#1}}
\newcommand{\cblue}[1]{{\color{blue}#1}}

\usepackage{hyperref}

\def\name{Vlad Predovic}

%% The following metadata will show up in the PDF properties
\hypersetup{
  colorlinks = true,
  urlcolor = black,
  pdfauthor = {\name},
  pdfkeywords = {cs344 ``operating systems'' files filesystem I/O},
  pdftitle = {Comparison of OS API's},
  pdfsubject = {CS 344 Final},
  pdfpagemode = UseNone
}

\parindent = 0.0 in
\parskip = 0.2 in

\begin{document}
\title{Comparison of selected POSIX and Windows API}
\subtitle{CS 344 Final, summer 2015 }
\author{Vlad Predovic}
\maketitle





\section{Introduction}
	Throughout this class several key topics of system programming have covered. Among these are file I/O, shell implementation with piping and redirection, 
distributing work through threading or spawning sub-processes, and cross language server/client interaction. However, 
if there is any key lesson to be learned about systems programming from this class it is not necessarily what can go right 
but just how quickly everything can go wrong if system level implementations are not handled with meticulous scrutiny(they mean the same thing).

	The following three sections provide a brief summary of the importance behind each tool. 
A comparison of differences in the API implementation in POSIX and windows will then be laid out.
Finally, these differences will be discussed, and, as in most duels to the death, a choice will be made for better implementation.
The topics covered will be socket implementation(specifically network sockets), file I/O, and communication through Piping.



\section{Sockets:}
	Sockets were first implemented in UNIX as a method allowing autonomous processes to communicate. Most socket implementations
at their core follow a client-server model where the server creates a socket that 'listens'
while the client creates a socket which attempts to to connect to the former. Ultimately, information can
be sent back and forth until the connection is interrupted.

\begin{enumerate}
  \item Set-up
Header files used: Windows(winsock.h)			POSIX(sys/socket.h, arpa/inet.h, netinet/in.h)

	For the most part, set-up of a simple socket connection is identical. Both implementations use the 'sockaddr_in'
struct to maintain addressing data for create the connection. However, the windows implementation requires a few extra lines
to initialize by using the WSAStartup() function. As a result, information on the WinSock implementation is stored in the WSADATA
struct and the function serves to initialize much of the data and structures that are necessary to use the sockets.


  \item Main Functionality
	Both the windows API and POSIX standard use the same basic methods and calls to communicate between
client and server.


  \item Clean up and Error Checking \ldots
	Since we initialize an additional struct we have to close it as well using WSAcleanup(). As for error 
checking, POSIX relies on the errno library to set flags and report while the windows API uses 
WSAgetlasterror() which reports the last error that occurred. Instead of directly closing the socket by
accesing the file descriptor using close(), we use the close-socket() function.
\end{enumerate}
	The changes are slight between the two implementations. A few different function calls are used to 
initialize, clean up, and error check functionality with basically the same parameters and functionality. These
basically consist of prefixing POSIX function call names with 'WSA'(ex: WSAsend). Windows is a little cleaner 
with only one necessary header file, and the WSAData struct can be ignored for the most part unless the user 
specifically wants information it contains. However, windows 8 introduced specific socket API's like WinRT 
and WebSockets, extensions used for specific purposes like high performance for developing network apps and special bi-directional web sockets.

Although very similar, Windows deserves a slight edge due to having additional and more updated functionality
which can be used for very specific purposes.


	The use of 
https://msdn.microsoft.com/en-us/library/ms740525(v=vs.85).aspx	
https://msdn.microsoft.com/en-us/library/windows/desktop/ms740642(v=VS.85).aspx
http://man7.org/linux/man-pages/man2/socket.2.html
\section{File I/O}
	One of the most critical aspects of an operating system is how objects interact with 
each other and the user. File I/O(input and output) refers to the operations used to access,
read, write, display, and manipulate standard files within the OS. 
	
\begin{enumerate}
  \item Header files used: Windows()			POSIX(sys/stat.h, fcntl.h, sys/types.h)
  
  \item Main Functionality
	OS functionality is to an extent defined by the fundamental needs and performance considerations 
a working system as we know it. Due to this, many similarities can be found between the operations
in the Windows and POSIX implementations. The CreateFile() call is the windows equivalent of the UNIX open()
call. It also returns a file descriptor as an integer used to refer to files such as sockets/terminals/etc..
In both versions a user must enter a pathname, access flags, and permissions. These similarites are seen throughout
most of the function calls, at least on a superficial levels. How both operating systems achieve the similar results 
tends to vary a little. 	

  \item File Descriptors(FD)
On both UNIX and Windows FD's map to their respective objects through a table.  
However, FD objects in UNIX can be sent via sockets while in windows they cannot. Windows instead
employs the function DuplicateHandle() which can be called by both the source and target processes.


  \item Clean up and Error Checking \ldots
http://man7.org/linux/man-pages/man2/open.2.html
\end{enumerate}

https://msdn.microsoft.com/en-us/library/windows/desktop/ms724251%28v=vs.85%29.aspx

\section{Piping}
\begin{enumerate}
  \item Set-up
Header files used: Windows()			POSIX()




  \item Main Functionality



  \item Clean up and Error Checking \ldots

\end{enumerate}


Redirecting: Used to pass output to a file. A redirect is an argument to a program.  What happens is that a program is ran and the output saved to a file. Then a second program  is ran using this temporary file as the input.

Piping: Used to pass output to a program or utility. s pipe seperates two commands.  An example would be a program being ran and all the output will be placed in the desired program, overwriting contents if they already exist.



\end{document}
