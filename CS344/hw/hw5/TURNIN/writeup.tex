\documentclass[letterpaper,10pt,titlepage]{article}

\usepackage{graphicx}

\usepackage{amssymb}
\usepackage{amsmath}
\usepackage{amsthm}

\usepackage{alltt}
\usepackage{float}
\usepackage{color}
\usepackage{verbatim}

\usepackage{geometry}
\geometry{textheight=10in, textwidth=7.5in}

\usepackage{hyperref}

\def\name{Vlad Predovic}

%pull in the necessary preamble matter for pygments output
\input{pygments.tex}

%% The following metadata will show up in the PDF properties
\hypersetup{
  colorlinks = true,
  urlcolor = black,
  pdfauthor = {\name},
  pdfkeywords = {cs311 ``operating systems'' sockets with JSON},
  pdftitle = {CS 311 Project 5:},
  pdfsubject = {CS 311 Project 5},
  pdfpagemode = UseNone
}

\parindent = 0.0 in
\parskip = 0.2 in

\begin{document}
\title{Sockets and Regular Expressions \\ CS 344 assignment 5, summer 2015}
\author{Vlad Predovic}
\maketitle
\section{Design}

\begin{itemize} 
\item The design of my program will begin by taking into consideration everything required from the algorithm in the compute.c file. 
Doing as much of compute.c first will help me to structure how I will need to send and recieve information.

\item For this assignment I really want to focus on testing everything locally before beginning to connect the many parts
of my project. For this reason I next plan on implementing the timing loop in my compute.c file before adding in communication.
I will insert rudimentary versions of the client/server examples into the manage/compute files respectively while building the other areas of the
assignment. In my compute.c file I plan on having a separate function to create each thread and compute the values autonomously. 
This function will then return to main, which is charge of gathering details like run time and killing if necessary.

\item Next I will look to implement a simple server on my manage.py file which upon completion, will be morphed
into a concurrent server. I plan on using the select method to handle multiple client connections at once. Creating a basic setup
should be simple enough. I just want the skeleton of what a concurrent server looks like before I implement the parsing aspect
of this assignment.

\item Once my compute.c is somewhat in order and I have a skeleton for my concurrent server I will work on passing information
back and forth through the system. This can be a bit tricky since network passes information as bytes and there is some header
space to deal with ass well. I will begin by just getting a simple parser moving back and forth and ultimately build up my JSON
parser.

\item After all this is completed I should be able to beef up my server and get it working with report.py. From what I understand
report.py should be fairly simple to implement as I am just calling for information.
\ldots 
\end{itemize}



\subsection{Where I differed}
For testing purposes I ended up using a server written in C, using many of the details provided by the walkthrough in the link below.
This was due to the fact that I programmed most compute.c first, except for the parsing and threading sections so I really wanted to understand
what was going on (networks and sockets are my favourite topic :D). 
Implementing a concurrent server was much more of an issue and complicated details. As a result I implemented most of the communication
between server and client first. I ended up brute-forcing the JSON due to a lack of time but it works well.




\section{Work log}
\verbatiminput{log.txt}

\begin{enumerate}
\item 8/4/2015 Tuesday:
	Began design of compute.c and implemented working brute force algorithm and simple timing loop during otherwise uneventful office hours.
\item 8/6/2015 Thursday:
	Read up on sockets, more design
\item 8/7/1015 Friday
	Implemented client-server system, implemented concurrency using children but discarded since I do not want to deal with children.
Hence the random babysitter function which should be renamed.
\item 8/12/2015 Wednesday 3:30am:
	One assignment, two final projects, a presentation, and an oral exam later I finally have time to work on this. 
Failed at implementing concurrent server using select method. Indentation issues in python proved to be my bane. I hate python.
\item 8/13/2015 Thursday 6:47pm:
	Working communication between server and client, JSON brute force works fairly well. Random start generator implemented.
Brb, I have another final to write up...
\item  8/13/2015 Thursday 11:25pm:
	Figured out my problem, I was using sendall instead of send throwing me into an infinite loop. Racing to implement as much
as I can now. Report.py works at least. 

\end{enumerate}
\section{Challenges}
I struggled with the python architecture in terms of implementing a concurrent server while trying to not break everything else. 
I feel this ultimately hindered me as I spent a lot of times debugging and changing the implementation of my server which took time 
away from working on the more difficult tasks like JSON parsing. 


\section{Answers}
\subsection{Main point of the assignment}
Understanding client server relationship and nitty gritty of how information is passed over a network. 
Also, improving string manipulation skills. Regular Expressions are pretty cool.

\subsection{Solution Accuracy}
The use of a client server system indirectly also serves as verification of  correctly working system.
This is because I am able to display information on separate sessions which would be very difficult
to arrive to accidentally. I used this to verify output on server side and client side.
Apart from this I used the GDB debugger during the PCRE implementation to follow variables '
and a massive amount of print statements. Also some error flags sprinkled throughout.


\subsection{Learnings}
Working with sockets was a lot of fun, in a masochistic sense of course, like everything else in systems programming.
Implementing how to properly send and receive information was very interesting as I had previously taken a
telecommunications class that was all theory(IE418).
Also became aware of the very interesting tool, regular expressions. Overall enjoyed this project.
I can really see the value in becoming fluent at using these regular expressions.

\end{document}